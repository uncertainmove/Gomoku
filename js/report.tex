\documentclass[utf8, cs4size]{ctexart}

\title{人工智能}
\author{曾圳}
\date{\today}

\usepackage[left=2.5cm, right=2.5cm, bottom=2.5cm]{geometry}
\usepackage{cite}
\usepackage[colorlinks, linkcolor=black, anchorcolor=black, citecolor=black]{hyperref}

\begin{document}

\maketitle

\section{Abstract}
\kaishu{
五子棋,也称五子连珠,是一个抽象的策略棋盘游戏。五子棋是世界智力运动会竞技项目之一,是一种两人对弈的纯策略型棋类游戏,通常双方分别使用黑白两色的棋子,下在棋盘直线与横线的交叉点上,先形成五子连线者获胜。\\
棋具与围棋通用,起源于中国上古时代的传统黑白棋一种。主要流行于华人和汉字文化圈的国家以及一些欧美地区,是世界上最古老的棋。\\
五子棋容易上手,老少皆宜,而且妙趣横生,引人入胜;不仅能增强思维能力,提高智力,而且富含哲理,有助于修身养性,已在各个游戏平台应用。\\
本次课设的目的是通过相关的策略实现一个智能AI,使其能和人类玩家进行博弈,并且具有一定的智慧。
}

\tableofcontents

\section{问题描述、知识表达}
\kaishu{
五子棋的棋盘采用的是传统围棋的棋盘,大小为 15 x 15,对弈方分执黑白两色棋子,按照黑子先行的规则交替在棋盘上放置棋子。当一方棋子在横线、竖线或是斜线上连续存在五颗时,则该方获得胜利。在该问题中未考虑禁手规则、交换规则等过于复杂的情况,不会对后手落子者进行任何补偿。该问题可以转化为深度优先搜索问题,即 AI 根据当前棋盘局势进行深度优先搜索,评估下次落子的最大收益位置。
}

\section{说明采用何种实现算法}
\kaishu{
本次课设采用了深度优先搜索算法,即 AI 遍历棋盘上所有可落子位置,在每个可落子位子模拟和对方进行对弈生成博弈树,找出收益最大的落子位置。如何计算该落子的收益呢?在这里提出一个假设,即博弈双方在下一次的落子上一定会使得此次落子的全局收益最大,所谓的全局收益就是此次落子之后,棋盘上己方的棋子得分减去对方的棋子得分。那么在深度优先搜索中,假设对弈双方为 AI 和 HUMAN,那么 AI 对于落子在点 (x, y) 的收益的计算方式为:AI 假设自己落子在点 (x, y),然后模拟 HUMAN 的落子,找出 HUMAN 的最佳落子位置,假设 HUMAN 在该位置落子的收益为 a ,则根据假设,我们认为 AI 落子在点 (x, y) 的收益为 -a。可以看到,在整个评估过程中会递归生成一棵树,该树为极大极小值生成树,通过该树可以选出最佳落子位置。在生成该树的过程中会涉及大量的模拟落子之后的棋盘收益计算,此时引入alpha-beta剪枝算法对该树的不必要分支进行裁剪。
}

\section{算法的原理、步骤、过程}
\kaishu{
本次课设的算法是深度优先搜索算法,在搜索过程中构建了极大极小值生成树,最后使用alpha-beta剪枝算法裁剪掉计算过程中的不必要的计算分支,加速计算。
}

\subsection{深度优先搜索算法}
\kaishu{

}

\subsection{极大极小值生成树}
\kaishu{

}

\subsection{alpha-beta剪枝}
\kaishu{

}

\section{代码实现}
\kaishu{
代码实现主要分为三个部分:UI实现,深度优先搜索算法(搜索过程中构建了带alpha-beta剪枝的极大极小值生成树)和全局收益评估函数。
}

\section{小结与展望}
\kaishu{

}
\end{document}